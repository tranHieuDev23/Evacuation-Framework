\documentclass{article}
\usepackage[utf8]{vietnam}
\usepackage{amsmath}
\usepackage{algorithm}
\usepackage[noend]{algpseudocode}
\usepackage{geometry}
\usepackage{float}

\makeatletter
\def\BState{\State\hskip-\ALG@thistlm}
\makeatother
\geometry{
		top = 35mm,
		left = 40mm,
		right = 30mm,
		bottom = 40mm
}

\begin{document}

\textsc{\LARGE Đại Học Bách Khoa Hà Nội} \\[1,5cm]
\textsc{\large Lớp Kĩ Sư Tài Năng - Công Nghệ Thông Tin K62} \\[0.5cm]
{\huge\bfseries Tìm đường thoát hiểm }\\[0.4cm] 

    \begin{flushleft}
        \large
        Ngày 8 tháng 5 năm 2019 \\[0.5cm]
        \textit{Author}\\[0,3cm] 
           
    \end{flushleft}
    
    \section{Tiêu chuẩn đề đánh giá trọng số con đường}
    Xét đoạn đường với chiều dài L, chiều ngang H, chỉ số an toàn T,
    và hiện tại có N người trên đoạn đường đó. Giả sử trong điều 
    kiện bình thường, đoạn đường không có người, độ an toàn T = 1, 
    thì với một người di chuyển vào con đường đó sẽ có vận tốc là V.
    Nhưng với ngoại cảnh cụ thể, thì vận tốc người đó sẽ nhỏ hơn,
    có thể coi như vận tốc thực tế của người đó tỉ lệ thuận với độ 
    an toàn và tỉ lệ nghịch với mật độ người trên con đường đó.
    Ta có thể xét 
    \begin{equation*}
        \emph{v} = \frac{V}{F(T, D)}
    \end{equation*} 
    $D = \frac{L*H}{X*Y}$ là mật độ người trên đoạn đường,
    $F(T, D)$ ta gọi là hàm ngữ cảnh, đặc trưng cho sự ảnh hưởng
    của ngoại cảnh tới vận tốc di chuyển của con người.
    Như vậy, thời gian để những người đó đi hết đoạn đường là:
    \begin{displaymath}
        t = \frac{L}{\emph{v}}  
        \Leftrightarrow t = \frac{L*F(T, D)}{V}   
    \end{displaymath}
    Đặt trọng số $W = \L*F(T, D)$ thì $t = \frac{W}{V}$. \\
    Như vậy, trong công thức trên, nếu V là vận tốc chuẩn của một 
    người thì w giống như chuẩn độ dài của con đường trong ngoại 
    cảnh cụ thể, và thời gian đi qua con đường sẽ được tính bởi 
    hai đơn vị trên. \\
    Ta sẽ giả sử tất cả mọi người trong tòa nhà đều có cùng vận tốc
    trung bình là $v_{tb}$, và khi đó thì để so sánh xem hai con
    đường nào tốt hơn, tức là ta phải so sánh thời gian thoát hiểm 
    của hai con đường đó, đồng nghĩa với việc ta phải so sánh trọng 
    số w của quãng đường. \\ 
    Ở đây, trong mô phỏng, ta sử dùng hàm:
    \begin{equation}
        F(T, D) = \frac{1}{T * (1.0001 - D)}
    \end{equation}
    
    \section{Ý tưởng thuật toán}
    Xét đoạn đường đi từ một Indicator u trong tòa nhà tới một exit 
    node e, gồm các corridor theo thứ tự là $p = <p_{1}, p_{2}, ...,p_{k}
    >$. Với mỗi corridor $p_{i}$ sẽ có tương ững $n_{i}$ người. \\ 
    Trọng số w của đoạn đường được tính: $w$ = $\sum{w_{i}}$ với 
    $w_{i} = \frac{L_{i}*g(D_{i})}{f(T_{i})}$ là trọng số của quãng 
    đường tương ứng tại thời điểm hiện tại. Với công thức như trên 
    thì ta thấy trọng số của đường đi từ Indicator u sẽ phụ thuộc vào mật độ lượng
    người trên mỗi cạnh tại thời điểm tính, giả sử là $t_{1}$.
    Ta có nhận xét rằng khi mà nhóm người A ở cạnh $p_{i1}$ tới 
    được được cạnh $p_{i2}$ (i2 > i1) thì những người ở cạnh 
    $p_{i2}$ đã di chuyển sang cạnh khác, tức là họ không ảnh hưởng 
    tới thời gian di chuyển của nhóm người A trên cạnh $p_{i2}$. Những 
    thành phần ảnh hưởng trực tiếp tới thời gian di chuyển trên cạnh 
    $p_{i2}$ của nhóm người A là điều kiện môi trường và lượng người 
    di chuyển trên đó cùng thời điểm nhóm A có di chuyển trên $p_{i2}$. 
    Như vậy, nếu ta muốn 
    lấy trọng số w để đặc trưng cho thời gian thoát hiểm của một
    nhóm người trên một đoạn đường thì không thể tính bởi công thức 
    như trên. Để có thể tính chi li, xét dãy $t = <t_{1}, t_{2},
    ..., t_{k}>$, $t_{i}$ là thời điểm mà số người đó bắt đầu đi vào
    quãng đường $p_{i}$, thì trọng số w sẽ được tính 
    \begin{equation}
        w = \sum{w_{i}}
    \end{equation}  
    với $w_{i}$ là trọng số của quãng đường $p_{i}$ tại thời điểm 
    $t_{i}$. \\
    Vấn đề là với mỗi đường đi khác nhau sẽ có cách chia thời gian 
    khác nhau, và để tính toán với từng đoạn đường như vậy là không 
    thể về khía cạnh thời gian tính toán. \\ 
    Với ý tưởng tương tự trên, ta sẽ chia đường đi làm 2 phần. 
    Phần đầu tiên là quãng đường đi được trong t(s) tiếp theo, 
    phần tiếp theo là đoạn còn lại. Giả sử trong t(s) tiếp theo 
    thì những người ở Indicator u đi đến được đoạn $p_{i}$. \\
    Đặt path1 = $<p_{1}, ..., p_{i-1}>$, path2 = $<p_{i}, ..., p_{k}
    >$. Trọng số của đoạn đường p sẽ được tính: 
    \begin{equation}
        w = w1 + w2 
    \end{equation}
    với \begin{itemize}
        \item $w1 = \sum{w_{j}}, j < i$, các đoạn $w_{j}$ được
        tính với số người trên đoạn là $n_{1}$ người, hay là số 
        người đang ở $p_{1}$ tại thời điểm hiện tại.
        \item $w2 = \sum{w_{j}}, i <= j <= k$, các đoạn $w_{j}$
        được tính với số người tới được $p_{i}$ trong t giây 
        tiếp theo.
    \end{itemize}
    \subsection{Mô hình tòa nhà}
    Tòa nhà sẽ có mô hình giống như mô hình tòa nhà trong thuật toán 
    LCDT, gồm các Indicator nhận thông tin từ máy chủ của toà nhà 
    và chỉ hướng cho những người gần đó trong trường hợp di tản. Nó 
    sẽ vẫn chứa dữ liệu backup dùng cho trường hợp mất kết nối với máy 
    chủ, sẽ là hướng chỉ đến đỉnh tiếp theo trong cây khung đường đi 
    ngắn nhất đến các lối ra. 

    \section{Nội dung thuật toán}
    \subsection{Mô hình chung}
    Để tiện cho việc tính toán, ta coi các Indicator như là các Node 
    trong đồ thị, nối các Corridor được biểu diễn như các Edge nối giữa
    các Node. \\ 
    Ta áp dụng thuật toán Dijkstra để tìm đường đi có trọng số nhỏ nhất 
    của mối node tới exitNode- lối ra của tòa nhà. Ta đưa ra hàm tính 
    trọng số của một cạnh như sau: 
    \begin{equation}
        W(edge) = L * F(edge.trustiness, edge.density) = \frac{L}{T * (1.01 - D)}
    \end{equation}
    Xét với một nhóm người A trên một cạnh \emph{e} đang đi trên đoạn đường path từ
    đó về tới exit node.
    Với ý tưởng chia đoạn đường từ đỉnh \emph{e} đến exit node thành hai đoạn, path1 và
    path2, trong đó: 
    \begin{itemize}
        \item path1: là đoạn đường gồm các đỉnh mà nhóm người A đi được trong thời gian 
        $\tau$. Xét S là đỉnh cuối cùng trong path1, tức là S là đỉnh xa nhất mà nhóm 
        người A đi được trong thời gian $\tau$ trên path. Ta xét rằng khi nhóm người A
        đi đến các đoạn đường P tiếp theo trên path1 thì những những người trước đó ở P
        đã di chuyển sang con đường khác. Như vậy thì các yêu tố ảnh hưởng tới thời gian 
        di chuyển của những người A trên path1 chỉ là các yếu tố vật lý, độ tin cậy và
        số lượng người trong nhóm A di chuyển trên đường. Do đó ta sẽ tính trọng số của 
        đoạn đường path1 tương ứng với số lượng người A, chứ không phải là số lượng người 
        trên cạnh của từng đoạn đường tại thời điểm tính toán.
        \item path2: là đoạn đường bắt đầu từ đỉnh S tới exit node trên con đường path.
        Khi nhóm người A di chuyển trên path1, thì những nhóm người khác ở các đỉnh xung 
        quanh cũng đang di chuyển. Như vậy khi nhóm người A tới được S thì cũng sẽ có các
        nhóm người khác tới được đỉnh S cùng thời điểm với nhóm người A. Như vậy từ đỉnh S,
        cũng với lập luận rằng những người ở đoạn đường phía sau đoạn chứa nhóm người A
        sẽ di chuyển đến các đoạn tiếp theo khi A tới được đó, như vậy thì
        lượng người trên các đoạn đường tiếp theo dọc theo path mà vào các thời điểm 
        nhóm người A đi qua thì sẽ có lượng người trên đó bằng số lượng người tới được 
        đỉnh S trong thời gian $\tau$. Do đó ta sẽ tính trọng số của đoạn đường path2 
        cho lượng người A theo công thức trước đó với số lượng người trên mỗi đoạn là 
        lượng người đến được S trong thời gian $\tau$.
    \end{itemize}
    Một điều quan trọng trong cách tính trọng số trên là làm làm sao biết được cáo bao 
    nhiêu người tại đỉnh S sau $\tau$ giây tiếp theo.  \\ 
    Trong thuật toán dijkstra, các đỉnh được gán nhãn sau là các đỉnh có trọng số đường đi 
    ngắn nhất tới đỉnh nguồn lớn hơn với các đỉnh được gán nhãn trước đó. Với công thức 
    tính trọng số đoạn đường mà đặc trưng được cho thời gian để con người đi hết đoạn đường
    đó thì những đỉnh nào được gán nhãn trước tức là những người ở gần đó sẽ tới exit 
    nhanh hơn, hay có nghĩa là việc họ di chuyển sẽ không ảnh hưởng tới những người 
    ở dằng sau khá xa họ. Và trong thuật toán này, khi một đỉnh đã tìm được đường đi 
    ngắn nhất cho nó, thì nó sẽ thông báo cho các đỉnh khác mà vẫn đang tìm đường biết
    được quãng đường của nó, để các đỉnh khác có thể tính toán lượng người cùng di chuyển 
    trong tương lai.

    
    \textbf{Node}: 
    \begin{itemize}
        \item \textit{next}: đỉnh được lựa chọn là hướng đi đến.
        \item \textit{nextEdge}: cạnh được lựa chọn là hướng đi đến.
        \item \textit{adjacences}: Danh sách struct gồm có:
        \begin{itemize}
            \item \textit{node}: đỉnh kề
            \item \textit{edge}: cạnh kề
            \item \textit{passingWeight}: trọng số của path mà đi qua đỉnh kề 
            \item \textit{reaching}: đỉnh tới được sau t(s) mà đi qua đỉnh 
            kề
        \end{itemize}    
        \item \textit{weight}: trọng số của đường đi tới root.
        \item \textit{nComingPeople}: số người sẽ đến sau t(s).
        \item \textit{tReachedNode}: đỉnh sẽ tới được sau t(s).
        \item \textit{tComingNodes}: danh sách đỉnh tới được sau t(s).
        \item \textit{label}: nhãn của đỉnh.
    \end{itemize}
    \textbf{Edge}
    \begin{itemize}
        \item \textit{to}: đỉnh tới.
        \item \textit{Length, Width, Trustness}: các thông số.
        \item \textit{nPeople}: số người trên cạnh.
        \item \textit{density}: mật độ.
        \item \textit{weight}: trọng số của cạnh.
    \end{itemize}
    \begin{algorithm}[!h]
    \caption{Algorithm caption}
    \label{alg:algorithm-label}
    \textbf{Input:} Đồ thị có trọng số ứng với các Indicator và Node\\
    \textbf{Output:} Đường đi ngắn nhất từ root đến mọi đỉnh trong đồ thị\\
    heap: cấu trúc heap min với phần tử là các đỉnh và so sánh dựa 
    trên weight.
    \begin{algorithmic}[1]
        \em
    \Procedure{Main Algorithm}{}
        \State heap.push(root)
        \State root.label $\gets$ true
        \While{$heap.size > 0$}
        \State $u \gets heap.pop()$
        \State u.label = true
        \State s $\gets$ u.tReachedNode
        \State s.nComingPeople $\gets$ s.nComingPeople + u.nextEdge.nPeople
        \State s.tComingNodes $\gets$ u
        \State \textbf{UpdateComingNode}(s)
        \For{v \textbf{in} u.adjacences \textbf{and} v.node.label = true}
        \State \textbf{UpdateComingPeople}(v.edge)
        \EndFor
        \For{v \textbf{in} u.adjacences \textbf{and} v.node.label = false} 
        \State s = \textbf{FindCrossNode}(v.node, v.edge)
        \State s.nComingPeople $\gets$ s.nComingPeople + v.edge.nPeople
        \State w1 $\gets$ \textbf{CalculateWeight}(u, s, v.edge.nPeople)
        \State w2 $\gets$ \textbf{CalculateWeight}(s, root, s.nComingPeople)
        \State newW $\gets$ v.edge.weight + w1 + w2
        \State ad \textbf{in} v.node.adjacences \textbf{which} ad.node = u
        \State ad.edgeWeight $\gets$ newW
        \State ad.reachedNode $\gets$ s
        \If{newW < v.node.weight}
        \State v.node.weight $\gets$ newW
        \State v.node.next $\gets$ u 
        \State v.node.tReachedNode $\gets$ s
        \State heap.push(v.node)
        \EndIf
        \State \textbf{end if}
        \State s.nComingPeople $\gets$ s.nComingPeople - E[v.node,u].nPeople
        \EndFor
        \textbf{end for}
        \EndWhile
        \textbf{end while}
    \EndProcedure
    \end{algorithmic}
    \end{algorithm}

    Trong thuật toán 1, từ dòng 7-10 thể hiện rằng khi đỉnh u được gán nhãn, thì nó 
    sẽ cập nhật cho đỉnh biết rằng nó sẽ tới được đó trong $\tau$ giây nữa. Sau khi 
    cập nhật thông tin cho đỉnh S thì đỉnh S có nhiệm vụ báo lại cho những đỉnh đằng 
    trước mà đang muốn đến S biết các thông tin để có thể tính toán lại trọng số khi 
    mà chọn con đường đi qua S( dòng 10).

    \begin{algorithm}
        \caption{UpdateComingPeople}
        \textbf{Input}:Edge edge, Node node \\ 
        Cạnh nằm giữa hai đỉnh đã được gán nhãn và chưa được cập nhật. \\
        \textbf{Output:}: Cập nhật số người ở cạnh đó cho đỉnh nó tới được sau t(s)

        \begin{algorithmic}
            \Procedure{My Procedure}{}
            \em
            \State s $\gets$ \textbf{FindCrossNode}(node, edge)
            \State s.nComingPeople $\gets$ s.nComingPeople + edge.nPeople
            \State \textbf{UpdateComingNode}(s)
            \EndProcedure
        \end{algorithmic}
    \end{algorithm}
    
    \begin{algorithm}
        \caption{FindCrossNode}
        \textbf{Input}: \textit{Node} node, \textit{Edge} edge\\
        \textbf{Output}: Trả lại đỉnh xa nhất trên đường đi ngắn nhất của u mà v 
        đi tới được trong thời gian t.
        \begin{algorithmic}
            \Procedure{My Procedure}{}
            \em
            \State sumWeight $\gets$ $v_{tb}*t$ 
            \State nPeople $\gets$ edge.nPeople
            \State from $\gets$ node
            \While{sumWeight > 0}
            \State sumWeight $\gets$ sumWeight - \textbf{GetWeight}(edge, nPeople)
            \State from $\gets$ edge.to
            \State edge $\gets$ from.nextEdge
            \If{from = root} break
            \EndIf
            \EndWhile 
            \textbf{end while} \\
            \Return from
            \EndProcedure
        \end{algorithmic}
    \end{algorithm}

    \begin{algorithm}
        \caption{CalculateWeight}
        \textbf{Input}: \textit{Node} u, \textit{Node} s, int nPeople \\
        \textbf{Output}: Trả lại trọng số của quãng đường đi từ u tới s với 
        nPeople người đi.
        \begin{algorithmic}
            \Procedure{My Procedure}{}
            \em
            \State weight $\gets$ 0
            \While{u \textbf{!=} s}
            \State edge $\gets$ u.nextEdge
            \State u $\gets$ u.next
            \State density $\gets$ \textbf{GetDensity}(edge, nPeople)
            \State weight $\gets$ weight + L * \textbf{F}(edge.Trustness, density)
            \EndWhile
            \Return weight
            \EndProcedure
        \end{algorithmic}
    \end{algorithm}

    \begin{algorithm}
        \caption{UpdateComingNode}
        \textbf{Input}: \textit{Node} s \\ 
        \textbf{Output}: Cập nhật lại trọng số của các Node mà sẽ tới được s
        trong thời gian t những vẫn chưa được gán nhãn.
        \begin{algorithmic}
            \Procedure{My Procedure}{}
            \For{u \textbf{in} s.tComingNodes \textbf{and} u.label = false}
            \State v \textbf{in} u.adjacences \textbf{which} v.reaching = s
            \State w1 $\gets$ \textbf{CalculateWeight}(v.node, s, v.edge.nPeople)
            \State w2 $\gets$ \textbf{CalculateWeight}(s, root, s.nComingPeople)
            \State v.passingWeight $\gets$ v.edge.weight + w1 + w2
            \State \textbf{GetNextNode}(u)
            \State heap.Rebuild(u)
            \EndFor
            \textbf{end for}
            \EndProcedure

            \Procedure{GetNextNode}{u}
            \Comment{Đặt lại đỉnh kề tốt nhất cho u}
            \For{v \textbf{in} u.adjacences \textbf{and} v.node.label = true}
            \If{v.edgeWeight < u.weight}
            \State u.weight = v.edgeWeight
            \State u.next = v.node
            \State u.tReachedNode = v.reachedNode \\
            \EndIf
            \textbf{end if}
            \EndFor
            \EndProcedure
        \end{algorithmic}
    \end{algorithm}
\end{document}